\documentclass{article}

\usepackage[english]{babel}

\usepackage[letterpaper,top=2cm,bottom=2cm,left=3cm,right=3cm,marginparwidth=1.75cm]{geometry}

\usepackage{amsmath}
\usepackage{graphicx}
\usepackage[colorlinks=true, allcolors=blue]{hyperref}
\usepackage{natbib}
\bibliographystyle{alpha}
\usepackage{caption}
\usepackage{float}

\title{Redes de Computadores \\ \large Trabalho Prático 1}
\author{Luís Felipe Ramos Ferreira}
\date{\href{mailto:lframos\_ferreira@outlook.com}{\texttt{lframos\_ferreira@outlook.com}}
}

\begin{document}

\maketitle

\section{Introdução}

O Trabalho Prático 2 da disciplina de Redes de Computadores teve como proposta
o desenvolvimento de um \textit{Blog} que permite a interação entre vários
clientes em um servidor usando sockets em linguagem C.

O repositório onde está armazenado o código utilizado durante o desenvolvimento
desse projeto
pode ser encontrado \href{https://github.com/lframosferreira/tp2-redes}{neste
      endereço}.

\section{Implementação}

Conforme especificado no enunciado, o projeto foi todo desenvolvido na
linguagem de programação C em um ambiente \textit{Linux}, e o manuseio de
sockets por
meio da interface POSIX
disponibilizada para a linguagem. Para manter uma maior organização do código,
além dos arquivos \textit{server.c} e \textit{client.c}, os quais possuem
respectivamente as implementações do servidor e do cliente,
foi criado um arquivo auxiliar \textit{common.c} e seu arquivo de cabeçalho
\textit{common.h}, os quais possuem as especificações e implementações de
funções auxiliares que podem ser utilizadas por ambas as partes do projeto.

\section{Desafios, dificuldades e imprevistos}

A primeira dificuldade imposta pelo trabalho prático foi a familiarização com a
interface POSIX de programação em sockets. Embora a linguagem C seja
considerada de alto nível, em muitos momentos suas funcionalidades podem ser

\section{Conclusão}

Em suma o projeto permitiu grandes aprendizados tanto na parte teórica como na

\section{Referências}

\begin{itemize}
      \item Livros:
            \begin{itemize}
                  \item Tanenbaum, A. S. \& Wetherall, D. (2011), Computer
                        Networks, Prentice Hall, Boston.
                  \item TCP/IP Sockets in C\@: Practical Guide for Programmers,
                        Second Edition
            \end{itemize}

      \item Web:
            \begin{itemize}
                  \item

                        \url{https://www.ibm.com/docs/en/zos/2.3.0?topic=sockets-using-sendto-recvfrom-calls}
                  \item

                        \url{https://www.educative.io/answers/how-to-implement-tcp-sockets-in-c}

            \end{itemize}

      \item Youtube:
            \begin{itemize}
                  \item \href{https://www.youtube.com/@JacobSorber}{Jacob
                              Sorber}
                  \item
                        \href{https://www.youtube.com/watch?v=Y6pFtgRdUts}{Jacob Sorber specific on
                              \textit{select}}
                  \item

                        \href{https://www.youtube.com/watch?v=_lQ-3S4fJ0U&list=PLPyaR5G9aNDvs6TtdpLcVO43_jvxp4emI}{Think
                              and Learn sockets playlist}
                  \item

                        \href{https://www.youtube.com/watch?v=tJ3qNtv0HVs&t=2s}{Playlist do professor
                              Ítalo Cunha}
            \end{itemize}

\end{itemize}

\end{document}