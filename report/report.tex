\documentclass{article}

\usepackage[english]{babel}

\usepackage[letterpaper,top=2cm,bottom=2cm,left=3cm,right=3cm,marginparwidth=1.75cm]{geometry}

\usepackage{amsmath}
\usepackage{graphicx}
\usepackage[colorlinks=true, allcolors=blue]{hyperref}
\usepackage{natbib}
\bibliographystyle{alpha}
\usepackage{caption}
\usepackage{float}

\title{Redes de Computadores \\ \large Trabalho Prático 1}
\author{Luís Felipe Ramos Ferreira}
\date{\href{mailto:lframos\_ferreira@outlook.com}{\texttt{lframos\_ferreira@outlook.com}}
}

\begin{document}

\maketitle

\section{Introdução}

O Trabalho Prático 2 da disciplina de Redes de Computadores teve como proposta
o desenvolvimento de um \textit{Blog} que permite a interação entre vários
clientes em um servidor usando \textit{sockets} e \textit{threads} com a
linguagem C.

O repositório onde está armazenado o código utilizado durante o desenvolvimento
desse projeto
pode ser encontrado \href{https://github.com/lframosferreira/tp2-redes}{neste
      endereço}.

\section{Implementação}

Conforme especificado no enunciado, o projeto foi todo desenvolvido na
linguagem de programação C em um ambiente \textit{Linux}, e o manuseio de
\textit{sockets} e \textit{threads} por
meio da interface POSIX
disponibilizada para a linguagem. Para manter uma maior organização do código,
além dos arquivos \textit{server.c} e \textit{client.c}, os quais possuem
respectivamente as implementações do servidor e do cliente,
foram criados uos arquivos auxiliares \textit{common.c} e seu arquivo de
cabeçalho
\textit{common.h}, os quais possuem as especificações e implementações de
funções auxiliares que podem ser utilizadas por ambas as partes do projeto,
assim como o arquivo \textit{topic.c} e seu arquivo de cabeçalho
\textit{topic.h},
os quais possuem as especificações e implementações relativas à estrutura de
dados que armazena os tópicos criados para o \textit{Blog}.

\subsection{Arquitetura do sistema}

Conforme especificado, o servidor implementado deve ser capaz de lidar com
múltiplas conexões de clientes ao mesmo tempo. Para realizar essa tarefa,
foram utilizadas \textit{threads} da interface POSIX.\@ Cada cliente possui um
\textit{thread} associada a ele no servidor, a qual está desanexada da
\textit{thread}
principal do programa. A escolha de implementá-las dessa maneira veio do fato
de que não há a necessidade de se realizar um \textit{join} em cada
\textit{thread} após que ela finalize sua execução, e assim
que a conexão for finalizada os recursos associados à aquela \textit{thread} em
específico podem ser desalocados.

Para armazenar os tópicos criados pelos clientes, foi criada uma
\href{https://en.wikipedia.org/wiki/Linked_list}{lista encadeada}, uma vez que
se trata de uma estrutura simples
de se criar e manipular, assim como cresce dinamicamente de forma controlada, o
que é desejável já que o número de tópicos que os clientes irão criar não é
determinado a princípio.

Para manter controle do menor identificador disponível para ser atribuído à um
novo cliente, um simples vetor binário que contêm um mapeamento de quais
identificadores estão sendo utilizados
é armazenado no servidor e este manipulado por uma função específica
denominada \textit{test\_and\_set\_lowest\_id()}.

Pelo lado do cliente, duas \textit{threads} são utilizadas. A \textit{thread}
principal, inicialmente, estabelece a primeira conexão com o servidor e obtêm o
identificador que foi atribuído pelo servidor à ele. Em seguida, ela irá
continuamente ler da entrada do usuário para saber qual a próxima operação que
ele deseja que seja executado.

A outra \textit{thread} é responsável por continuamente escutar o servidor e
imprimir em tela o que for necessário. Por exemplo, nos tipos de operação
\textit{list topics} e \textit{publish in <topic>}.

Para que cada \textit{thread} do servidor saiba para quais \textit{sockets} ele
deve enviar a mensagem de adição de um novo tópico, após uma ação desse tipo, o
descritor de arquivo do \textit{socket} de cada cliente é armazenado juntamente
na lista de clientes conectados, facilitadno assim a troca de mensagens na
arquitetura do sistema.

\subsection{Condições de corrida}

Por se tratar de uma arquitetura com múltiplas \textit{threads}, problemas
como
\href{https://learn.microsoft.com/en-us/troubleshoot/developer/visualstudio/visual-basic/language-compilers/race-conditions-deadlocks}{condições
      de corrida}
podem acontecer. Uma vez que estruturas como a lista encadeada de tópicos e o
vetor de identificadores disponíveis
podem ser acessados e manipulados por diferentes \textit{threads} ao mesmo
tempo, situações estranhas e não determinísticas podem ocorrer.

Para resolver esse problema, foi adotado o uso da estrutura de um
\href{https://www.ibm.com/docs/pt-br/aix/7.3?topic=programming-using-mutexes}{mutex},
que garante que duas threads não irão acessar/modificar a mesma estrutura de
dados global
ao mesmo tempo.

\section{Desafios, dificuldades e imprevistos}

A primeira dificuldade imposta pelo Trabalho Prático II foi a familiarização
com a
interface POSIX de programação em \textit{threads}. Implementar uma arquitetura
com múltiplas \textit{threads}
nunca é trivial, e garantir que condições de corridas não irão ocorrer e que a
memória alocada para as \textit{threads} será desalocada assim que elas não
forem mais utilizadas foi um grande empecilho. Como comentado anteriormente,
evitar problemas como condições de corrida foi feita por meio da implementação
de vários \textit{mutex}.

O principal desafio do trabalho, no entanto, se referiu à como fazer a operação
de \textit{publish in <topic>} funcionar. Quando um cliente executa essa ação,
uma mensagem de aviso deve ser redirecionada para todos os outros
clientes inscritos naquele tópico em particular, e compreender como fazer esse
tipo de comunicação tomou muito tempo. A solução adotada segue a seguinte
estrutura:

\begin{enumerate}
      \item O servidor possui \(N\) \textit{threads}, onde \(N - 1\) são
            utilizadas para manter a conexão com cada um dos \(N - 1\) clientes
            atualmente
            conectados e a outra responsável por escutar e aceitar novas
            conexões.
      \item O cliente possui duas \textit{threads}. Uma delas é responsável por
            ler da entrada do usuário e processar as operações desejadas,
            enquanto a outra
            constantemente escuta possíveis mensagens enviadas pelo servidor.
      \item O servidor mantêm uma lista com os descritores de arquivo dos
            \textit{sockets} atrelados a cada cliente conectado. Quando a
            operação de \textit{publish in <topic>} é processada, a lista de
            clientes
            inscritos naquele tópico é analisada e, para cada um deles, a
            \textit{thread} responsável pelo cliente que enviou a operação
            inicial envia
            para cada um desses outros clientes a mensagem de aviso de nova
            publicação.
\end{enumerate}

Essa solução é funcional e cumpriu os critérios estabelecidos, embora acredite
que existam soluçções mais complexas onde as próprias \textit{threads}
responsáveis por cada cliente consigam enviar as mensagens de aviso para seus
respectivos clientes por meio de algum tipo de comunicação entre
\textit{threads}.

\section{Conclusão}

Em suma o projeto permitiu grandes aprendizados tanto na parte teórica como na
parte prática no que se refere à programação em redes. Em particular, a
arquitetura necessária para esse projeto em particular possibilitou que
importantes conceitos ligados a programação com \textit{threads} fossem
revisitados.

Assim como o Trabalho Prático I, esse trabalho tornou compreender melhor como
funciona o protocolo de comunicação TCP, como
deve ser feita e mantida a comunicação entre um servidor e um cliente, etc.

\section{Referências}

\begin{itemize}
      \item Livros:
            \begin{itemize}
                  \item Tanenbaum, A. S. \& Wetherall, D. (2011), Computer
                        Networks, Prentice Hall, Boston.
                  \item TCP/IP Sockets in C\@: Practical Guide for Programmers,
                        Second Edition
            \end{itemize}

      \item Web:
            \begin{itemize}
                  \item

                        \url{https://www.ibm.com/docs/en/zos/2.3.0?topic=sockets-using-sendto-recvfrom-calls}
                  \item

                        \url{https://www.educative.io/answers/how-to-implement-tcp-sockets-in-c}

            \end{itemize}

      \item Youtube:
            \begin{itemize}
                  \item \href{https://www.youtube.com/@JacobSorber}{Jacob
                              Sorber}
                  \item

                        \href{https://www.youtube.com/watch?v=_lQ-3S4fJ0U&list=PLPyaR5G9aNDvs6TtdpLcVO43_jvxp4emI}{Think
                              and Learn sockets playlist}
                  \item

                        \href{https://www.youtube.com/watch?v=tJ3qNtv0HVs&t=2s}{Playlist do professor
                              Ítalo Cunha}
            \end{itemize}

\end{itemize}

\end{document}